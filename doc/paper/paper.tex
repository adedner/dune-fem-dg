\documentclass{ansarticle}

\title{Preparing articles for submission to the \\
       Archive of Numerical Software}
\author[1]{Editor O. Archive\thanks{Additional thanks to my neighbor's dog
  for waking me up on time to work on this style file}}
\author[2]{Author O. Software}
\affil[1]{The Archive of Numerical Software}
\affil[2]{The Worlds Best Place for Numerical Software}
\runningtitle{Preparing articles for ANS}
\runningauthor{Archive Editors}

\usepackage{xspace}
\usepackage{bbm}

% definitions used by included articles, reproduced here for 
% educational benefit, and to minimize alterations needed to be made
% in developing this sample file.

\newcommand{\pe}{\psi}
\def\d{\delta} 
\def\ds{\displaystyle} 
\def\e{{\epsilon}} 
\def\eb{\bar{\eta}}  
\def\enorm#1{\|#1\|_2} 
\def\Fp{F^\prime}  
\def\fishpack{{FISHPACK}} 
\def\fortran{{FORTRAN}} 
\def\gmres{{GMRES}} 
\def\gmresm{{\rm GMRES($m$)}} 
\def\Kc{{\cal K}} 
\def\norm#1{\|#1\|} 
\def\wb{{\bar w}} 
\def\zb{{\bar z}} 

\newcommand{\phispace}{V_h}
\newcommand{\comment}[1]{}

% grid notation
\newcommand{\grid}{{\mathcal{T}_h}}
\newcommand{\entity}{K}
\newcommand{\elem}{\entity}
\newcommand{\isec}{e}
\newcommand{\maxnuminterfaces}{N_\grid}
\newcommand{\numinterfacesinflow}{N^{in}}
\newcommand{\numinterfacesoutflow}{N^{out}}
\newcommand{\maxnuminterfacesinflow}{N^{in}_\grid}
\newcommand{\maxnuminterfacesoutflow}{N^{out}_\grid}
%\newcommand{\RRR}{\mathbb{R}}
\newcommand{\RRR}{{\mathbbm R}}
\newcommand{\NNN}{{\mathbbm N}}


\newcommand{\maxnuminterface}{\mathcal{N}_\grid}


% flux function
\newcommand{\flux}[1]{\widehat{#1}}
\newcommand{\bndflux}[1]{\widehat{#1}}
\newcommand{\fluxF}{\flux{\mathcal{F}}_{\isec}}
\newcommand{\fluxu}{\flux{\mathcal{U}}_{\isec}}
\newcommand{\fluxA}{\flux{\mathcal{A}}_{\isec}}
\newcommand{\bndfluxf}{\bndflux{f}}
\newcommand{\bndfluxu}{\bndflux{u}}
\newcommand{\bndfluxA}{\bndflux{A}}

%\newcommand{\boldsymbol}[1]{\mathbf{#1}}

\newcommand{\vect}[1]{\boldsymbol{#1}}
\newcommand{\vecU}{\vect{U}}
\newcommand{\vecV}{\vect{V}}
\newcommand{\vecK}{\vect{K}}
\newcommand{\vecI}{\vect{I}}
\newcommand{\vecW}{\vect{W}}
\newcommand{\vecv}{\vect{v}}
\newcommand{\vecw}{\vect{w}}
\newcommand{\basefct}{\vect{\varphi}}
\newcommand{\vbold}{\boldsymbol{v}}
\newcommand{\nbold}{\boldsymbol{n}}
\newcommand{\taubold}{\boldsymbol{\tau}}
\newcommand{\df}{\vecU_h}

\newcommand{\ics}{\vect{x}}


% operators
\newcommand{\oper}[1]{\mathcal{#1}}
\newcommand{\spcoper}{\oper{L}_h}
\newcommand{\jump}[1]{[ {#1} ]_{\isec}  }
\newcommand{\vjump}[1]{[ \! [ {#1} ] \! ]_{\isec} }
\newcommand{\aver}[1]{\{ {#1} \}_{\isec} }
\newcommand{\vaver}[1]{\{ \! \! \{ {#1} \} \! \! \}_{\isec} }
\newcommand{\vswitchaver}[1]{\{ \! \! \{ {#1} \} \! \! \}_{\switch} }
\newcommand{\Au}{A(\df)}
\newcommand{\su}{S(\df)}

\newcommand{\dif}[1]{\partial_{#1}}

\newcommand{\liftr}{{\boldsymbol{r}}}
\newcommand{\liftre}{{\liftr_e}}
\newcommand{\liftfactor}{\chi}
\newcommand{\Kminus}{{K^-_e}}
\newcommand{\dune}{\textsc{Dune}\xspace}
\newcommand{\dunefem}{\textsc{Dune-Fem}\xspace}
\newcommand{\dunefemdg}{\textsc{Dune-Fem-DG}\xspace}
\newcommand{\alugrid}{\code{ALUGrid}\xspace}
\newcommand{\spgrid}{\code{SPGrid}\xspace}
\newcommand{\likwid}{\code{likwid}\xspace}
\newcommand{\code}[1]{\texttt{#1}}

% \newcommand{}{}
\newcommand{\trans}{\mathrm{T}}
\newcommand{\gasconst}{R_d}
\newcommand{\poldeg}{\text{k}}



%------------------------------------------------------------------------------
\begin{document}

\maketitle

\begin{abstract}
We discuss the matrix-free
implementation of Discontinuous Galerkin methods for compressible flow
problems, i.e. the compressible Navier-Stokes equations. For the spatial discretization
the CDG2 method and for temporal discretization an explicit Runge-Kutta method is used.
For the presented matrix-free approach we discuss asynchronous communication,
shared memory parallelization, and automated code generation to increase the
floating point performance of the code.
\end{abstract}

%------------------------------------------------------------------------------
%------------------------------------------------------------------------------
\section{Introduction}



%------------------------------------------------------------------------------
%------------------------------------------------------------------------------
\section{Governing Equations}
\label{sec:equations}
The system we investigate is governed by the
viscous compressible flow equations in $\theta$-form,
for example, described in \cite{GR08}.
For $\Omega \subset \RRR^d$, $d=1,2,3$, these equations can be written in the form
\begin{eqnarray}
\label{eqn:general}
\label{eqn:ns}
\partial_t \vecU  &=& \oper{L}(\vecU) \qquad \mbox{ in } (0,T] \times \Omega, \\
\vecU(0,\cdot) &=& \vecU_0(\cdot) \qquad  \mbox{ in } \Omega, \nonumber 
\end{eqnarray}
with $$\oper{L}(\vecU) := - \nabla \cdot\big( \mathcal{F}(\vecU) -
\mathcal{A}(\vecU,\nabla\vecU) \big) + \mathcal{S}(\vecU)$$
and suitable boundary conditions.

\todo{Here U should not be specified yet.}
The vector of conservative variables is $\vecU=(\rho, \rho\vecv, \rho\theta)^\trans$.
$\rho$ is the density, $\theta$ the potential temperature, and
$\vecv=(v_1,...,v_d)^T$ the velocity field.
$\mathcal{F}(\vecU) = (\mathcal{F}_i(\vecU))$ and
$\mathcal{A}(\vecU)\nabla\vecU = ((\mathcal{A}(\vecU)\nabla \vecU)_i )$ for $i=1,...,d$, are
given as follows:
\begin{equation}
   \mathcal{F}_i(\vecU)=\left(\begin{array}{c}
    \rho v_i \\
    \rho v_1 v_i  + \delta_{1i}p \\
    \vdots \\
    \rho v_d v_i + \delta_{di} p \\
    v_i \rho\theta  \\
  \end{array}\right),\qquad 
  \big (\mathcal{A}(\vecU)\nabla\vecU \big )_i =\mu\rho\left(\begin{array}{c}
    0                \\
    \partial_i v_1   \\ 
    \vdots           \\ 
    \partial_i v_d   \\
    \partial_i\theta \\
  \end{array}\right).
\end{equation}
with $\mu$ being the kinematic viscosity.
The source term $\mathcal{S}$ is only acting on the last component of the velocity field, i.e.
$\mathcal{S}(\vecU) = (0,...,0,-\rho g,0)^T$ with
$g$ being the constant of the gravitation force.
To close the system we define the pressure $p$ in accordance with the
ideal gas law $p = p_0\left(\frac{\rho\gasconst\theta}{p_0}\right)^\gamma$,
where $\gamma=c_p/c_v$ is the heat capacity ratio and $c_p$ and $c_v$ are specific
heat capacities under constant pressure and volume, respectively.
The individual gas constant is defined as $\gasconst=c_p-c_v$.
%In equation (\ref{eqn:eos}) 
%$p_0$ is the standard reference pressure,
For the standard reference pressure $p_0$ we choose $p_0=10^5$~Pa.


%------------------------------------------------------------------------------
%------------------------------------------------------------------------------
% Discretization
%%%%%%%%%%%%%%%%%%%%%%%%%%%%%%%%%%%%%%%%%%%%%%%
%%%%%%%%%%%%%%%%%%%%%%%%%%%%%%%%%%%%%%%%%%%%%%%

\section{Discretization}
\label{sec:dune}
\label{seq:discretization}

The considered discretization is based on the Discontinuous Galerkin (DG) approach and 
implemented in \dunefem (\cite{dunefempaper}) a module of the
\dune framework (\cite{dunepaperII:08,dunepaperI:08}).
The current state of development allows for simulation of convection dominated 
(cf.~\cite{limiter:11}) as well as viscous flow (cf.~\cite{cdg2:10}).
We consider the CDG2 method from 
(\cite{cdg2:10}) of up to 5th order in space and 3rd order in time for the numerical
investigations carried out in this paper.

\subsection{Spatial Discretization}

The spatial discretization is derived in the following way. 
Given a tessellation $\grid$ of the domain $\Omega$ with 
$\cup_{K \in \grid} K = \Omega$ the 
discrete solution $\df$ is sought in the piecewise polynomial space 
\begin{equation}
\label{eqn:vspace}
    \phispace = \{\vecv\in L^2(\Omega,\RRR^{d+2}) \; \colon
    \vecv|_{K}\in[\mathcal{P}_k(K)]^{d+2}, \ K\in\grid\}
      \quad\textrm{for some}\;k \in \NNN, \nonumber
\end{equation}
where $\mathcal{P}_k(K)$ is a space containing polynomials up to degree
$k$. On quadrilateral or hexahedral 
elements we replace $\mathcal{P}_k$ with $\mathcal{Q}_k$ build by products of Legendre polynomials
of up to degree $k$ in each coordinate.

\newcommand{\dual}[1]{\langle \basefct, #1 \rangle}% 
We denote with $\Gamma_i$ the set of all intersections between two 
elements of the grid $\grid$ and accordingly with $\Gamma$ the set of all
intersections, also with the boundary of the domain $\Omega$. 
The following discrete form is not the most general but still
covers a wide range of well established DG methods. 
For all basis functions $\basefct \in \phispace$ we 
define 
\begin{equation}
\label{convDiscr}
\dual { \spcoper(\df) } := \dual{ \oper{K}_h(\df) } + \dual{ \oper{I}_h(\df) }
\end{equation}
with the element integrals  
\begin{eqnarray}
\label{eqn:elementint}
   % \int_\Omega\partial_{t}\df\cdot\basefct \,dx &=&
   \dual{ \oper{K}_h(\df) } &:=&
      %\int_{\Omega} 
      \sum_{\elem \in \grid} \int_{\elem}
      \big( ( \mathcal{F}(\df) - \mathcal{A}(\vecU_h) \nabla \df ) : \nabla\basefct + \su
      \cdot \basefct \big),
\end{eqnarray}
and the surface integrals (by introducing appropriate numerical fluxes 
$\fluxF$, $\fluxA$ for the convection and diffusion terms, respectively) 
\begin{eqnarray}
\label{eqn:surfaceint}
   % \int_\Omega\partial_{t}\df\cdot\basefct \,dx &=&
   \dual{ \oper{I}_h(\df) } &:=&
      \sum_{e \in \Gamma_i} \int_e \big(
      \vaver{\mathcal{A}(\vecU_h)^T\nabla\basefct} : \vjump{\df} +
      \vaver{\mathcal{A}(\vecU_h)\nabla\df} : \vjump{\basefct} \big) \\
    &-& \sum_{e \in \Gamma} \int_e \big( \fluxF(\df) - \fluxA(\df)\big) :
      \vjump{\basefct},
      \nonumber 
\end{eqnarray}
where $\vaver{ \vecV } = \frac{1}{2}( \vecV^+ + \vecV^- )$ denotes the average and 
$\vjump{ \vecV } = (\nbold^+ \otimes \vecV^+  + \nbold^-\otimes \vecV^-) $ the jump of the
discontinuous function $\vecV\in V_h$ over element boundaries.
For matrices $\sigma,\tau\in\RRR^{m\times n}$ we use standard notation
$\sigma : \tau = \sum_{j=1}^m\sum_{l=1}^n\sigma_{jl}\tau_{jl}$. Additionally, for vectors
$\vecv \in \RRR^m,\vecw\in\RRR^n$, we define $\vecv\otimes\vecw\in\RRR^{m\times n}$
according to $(\vecv\otimes\vecw)_{jl}=\vecv_j \vecw_l$ for $1\leq j\leq m$, $1\leq l\leq n$.

The convective numerical flux $\fluxF$ can be any appropriate numerical flux known for
standard finite volume methods. 
For the results presented in this paper we choose $\fluxF$ to be the 
HLL numerical flux function described in~(\cite{ChefBuch}).

A wide range of diffusion fluxes $\fluxA$ can be found in the
literature, for a summary see (\cite{uni:02}).
We choose the CDG2 flux
\begin{eqnarray}
\fluxA(\vecV) := 2\liftfactor_e \big(\mathcal{A}(\vecV)\liftre(\vjump{\vecV})\big)|_{\Kminus}
\quad\mbox{for } \vecV\in V_h,
\end{eqnarray}
which was shown to be highly efficient for the Navier-Stokes equations (cf. \cite{cdg2:10}). 
Based on stability results, we choose $\Kminus$ to be the element adjacent to the edge $e$ with the smaller
volume. $\liftre(\vjump{\vecV})\in [\phispace]^d$ is the lifting of the jump of $\vecV$ defined by
\begin{eqnarray}
    \int_\Omega \liftre(\vjump{\vecV}) : \taubold = -\int_e
    \vjump{\vecV} : \vaver{\taubold} \quad
    \mbox{for all}\;\taubold\in [\phispace]^d.
\end{eqnarray}
For the numerical experiments done in this paper we use $\liftfactor_e= \frac{1}{2}\maxnuminterface$,
where $\maxnuminterface$ is the maximal number of intersections one element in the grid
can  have (cf. \cite{cdg2:10}). For most of the 
numerical results in this paper we use
conforming hexahedral elements and thus $\liftfactor_e=3$ for all $\isec \in \Gamma$.

%%%%%%%%%%%%%%%%%%%%%%%%%%%%%%%%%%%%%%%%%%%%%%%
%%%%%%%%%%%%%%%%%%%%%%%%%%%%%%%%%%%%%%%%%%%%%%%
\subsection{Temporal discretization}
\label{TimeDisc}

The discrete solution $\df(t) \in \phispace$ 
has the form $\df(t,x) = \sum_i \vecU_i(t)\basefct_i(x)$.
We get a system of ODEs for the coefficients of $\vecU(t)$ which reads 
\begin{eqnarray}
  \label{eqn:ode}
  \vecU'(t) &=& f(\vecU(t),t)  \mbox{ in } (0,T]
\end{eqnarray}
with $f(\vecU(t),t) = M^{-1}\spcoper(\df(t),t)$, $M$ being the mass matrix which is in
our case block diagonal or even the identity, depending on the choice of basis
functions. $\vecU(0)$ is given by the projection of $\vecU_0$ onto $\phispace$.

Disregarding the order of the spatial discretization
we use an explicit \textit{Strong Stability
Preserving} Runge-Kutta method (SSP-RK) of third order (\cite{shu:01}). 
Implicit or semi-implicit Runge-Kutta solvers based on a Jacobian-free Newton-Krylov
method (see \cite{knoll:04}) are also available for the proposed DG method 
and implemented in the \dunefem framework.
The results and implementation techniques 
presented in this paper can be applied to explicit, 
implicit, or semi-implicit methods, as long as a 
\textbf{matrix-free} implementation of the discrete operator 
$\spcoper$ is used.

%%%%%%%%%%%%%%%%%%%%%%%%%%%%%%%%%%%%%%%%%%%%%%%
%%%%%%%%%%%%%%%%%%%%%%%%%%%%%%%%%%%%%%%%%%%%%%%



%------------------------------------------------------------------------------
%------------------------------------------------------------------------------
% Implementation
\section{Implementation}
\subsection{Problem selector (rename?)}

\subsection{Problem}
\subsection{Model}
\subsection{Stepper}

\subsection{Adaptivity}
\subsection{Parallelization}
\subsection{I/O}

%------------------------------------------------------------------------------
%------------------------------------------------------------------------------
% Numerical Examples 
%%%%%%%%%%%%%%%%%%%%%%%%%%%%%%%%%%%%%%%%%%%%%%%%%%%%%%%%%%%
% Numerical examples
%%%%%%%%%%%%%%%%%%%%%%%%%%%%%%%%%%%%%%%%%%%%%%%%%%%%%%%%%%%
\section{Numerical examples}

\todo{Discuss Appearance}

\subsection{Advection-Diffusion equation}
\todo{Define U}

\subsection{Euler equation}
\todo{Define U}

\subsection{Compressible Navier-Stokes}
\todo{Define U}

\subsection{Poisson equation}
\todo{Define U}

\subsection{Stokes}
\todo{Define U and equations}

\subsection{Incompressbile Navier-StokesStokes}
\todo{Define U and equations}



\bibliographystyle{abbrvnat}
\bibliography{bibliography}
\end{document}
