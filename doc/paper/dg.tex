%%%%%%%%%%%%%%%%%%%%%%%%%%%%%%%%%%%%%%%%%%%%%%%
%%%%%%%%%%%%%%%%%%%%%%%%%%%%%%%%%%%%%%%%%%%%%%%

\section{Discretization}
\label{sec:dune}
\label{seq:discretization}

The considered discretization is based on the Discontinuous Galerkin (DG) approach and 
implemented in \dunefem (\cite{dunefempaper}) a module of the
\dune framework (\cite{dunepaperII:08,dunepaperI:08}).
The current state of development allows for simulation of convection dominated 
(cf.~\cite{limiter:11}) as well as viscous flow (cf.~\cite{cdg2:10}).
We consider the CDG2 method from 
(\cite{cdg2:10}) of up to 5th order in space and 3rd order in time for the numerical
investigations carried out in this paper.

\subsection{Spatial Discretization}

The spatial discretization is derived in the following way. 
Given a tessellation $\grid$ of the domain $\Omega$ with 
$\cup_{K \in \grid} K = \Omega$ the 
discrete solution $\df$ is sought in the piecewise polynomial space 
\begin{equation}
\label{eqn:vspace}
    \phispace = \{\vecv\in L^2(\Omega,\RRR^{d+2}) \; \colon
    \vecv|_{K}\in[\mathcal{P}_k(K)]^{d+2}, \ K\in\grid\}
      \quad\textrm{for some}\;k \in \NNN, \nonumber
\end{equation}
where $\mathcal{P}_k(K)$ is a space containing polynomials up to degree
$k$. On quadrilateral or hexahedral 
elements we replace $\mathcal{P}_k$ with $\mathcal{Q}_k$ build by products of Legendre polynomials
of up to degree $k$ in each coordinate.

\newcommand{\dual}[1]{\langle \basefct, #1 \rangle}% 
We denote with $\Gamma_i$ the set of all intersections between two 
elements of the grid $\grid$ and accordingly with $\Gamma$ the set of all
intersections, also with the boundary of the domain $\Omega$. 
The following discrete form is not the most general but still
covers a wide range of well established DG methods. 
For all basis functions $\basefct \in \phispace$ we 
define 
\begin{equation}
\label{convDiscr}
\dual { \spcoper(\df) } := \dual{ \oper{K}_h(\df) } + \dual{ \oper{I}_h(\df) }
\end{equation}
with the element integrals  
\begin{eqnarray}
\label{eqn:elementint}
   % \int_\Omega\partial_{t}\df\cdot\basefct \,dx &=&
   \dual{ \oper{K}_h(\df) } &:=&
      %\int_{\Omega} 
      \sum_{\elem \in \grid} \int_{\elem}
      \big( ( \mathcal{F}(\df) - \mathcal{A}(\vecU_h) \nabla \df ) : \nabla\basefct + \su
      \cdot \basefct \big),
\end{eqnarray}
and the surface integrals (by introducing appropriate numerical fluxes 
$\fluxF$, $\fluxA$ for the convection and diffusion terms, respectively) 
\begin{eqnarray}
\label{eqn:surfaceint}
   % \int_\Omega\partial_{t}\df\cdot\basefct \,dx &=&
   \dual{ \oper{I}_h(\df) } &:=&
      \sum_{e \in \Gamma_i} \int_e \big(
      \vaver{\mathcal{A}(\vecU_h)^T\nabla\basefct} : \vjump{\df} +
      \vaver{\mathcal{A}(\vecU_h)\nabla\df} : \vjump{\basefct} \big) \\
    &-& \sum_{e \in \Gamma} \int_e \big( \fluxF(\df) - \fluxA(\df)\big) :
      \vjump{\basefct},
      \nonumber 
\end{eqnarray}
where $\vaver{ \vecV } = \frac{1}{2}( \vecV^+ + \vecV^- )$ denotes the average and 
$\vjump{ \vecV } = (\nbold^+ \otimes \vecV^+  + \nbold^-\otimes \vecV^-) $ the jump of the
discontinuous function $\vecV\in V_h$ over element boundaries.
For matrices $\sigma,\tau\in\RRR^{m\times n}$ we use standard notation
$\sigma : \tau = \sum_{j=1}^m\sum_{l=1}^n\sigma_{jl}\tau_{jl}$. Additionally, for vectors
$\vecv \in \RRR^m,\vecw\in\RRR^n$, we define $\vecv\otimes\vecw\in\RRR^{m\times n}$
according to $(\vecv\otimes\vecw)_{jl}=\vecv_j \vecw_l$ for $1\leq j\leq m$, $1\leq l\leq n$.

The convective numerical flux $\fluxF$ can be any appropriate numerical flux known for
standard finite volume methods. 
For the results presented in this paper we choose $\fluxF$ to be the 
HLL numerical flux function described in~(\cite{ChefBuch}).

A wide range of diffusion fluxes $\fluxA$ can be found in the
literature, for a summary see (\cite{uni:02}).
We choose the CDG2 flux
\begin{eqnarray}
\fluxA(\vecV) := 2\liftfactor_e \big(\mathcal{A}(\vecV)\liftre(\vjump{\vecV})\big)|_{\Kminus}
\quad\mbox{for } \vecV\in V_h,
\end{eqnarray}
which was shown to be highly efficient for the Navier-Stokes equations (cf. \cite{cdg2:10}). 
Based on stability results, we choose $\Kminus$ to be the element adjacent to the edge $e$ with the smaller
volume. $\liftre(\vjump{\vecV})\in [\phispace]^d$ is the lifting of the jump of $\vecV$ defined by
\begin{eqnarray}
    \int_\Omega \liftre(\vjump{\vecV}) : \taubold = -\int_e
    \vjump{\vecV} : \vaver{\taubold} \quad
    \mbox{for all}\;\taubold\in [\phispace]^d.
\end{eqnarray}
For the numerical experiments done in this paper we use $\liftfactor_e= \frac{1}{2}\maxnuminterface$,
where $\maxnuminterface$ is the maximal number of intersections one element in the grid
can  have (cf. \cite{cdg2:10}). For most of the 
numerical results in this paper we use
conforming hexahedral elements and thus $\liftfactor_e=3$ for all $\isec \in \Gamma$.

%%%%%%%%%%%%%%%%%%%%%%%%%%%%%%%%%%%%%%%%%%%%%%%
%%%%%%%%%%%%%%%%%%%%%%%%%%%%%%%%%%%%%%%%%%%%%%%
\subsection{Temporal discretization}
\label{TimeDisc}

The discrete solution $\df(t) \in \phispace$ 
has the form $\df(t,x) = \sum_i \vecU_i(t)\basefct_i(x)$.
We get a system of ODEs for the coefficients of $\vecU(t)$ which reads 
\begin{eqnarray}
  \label{eqn:ode}
  \vecU'(t) &=& f(\vecU(t),t)  \mbox{ in } (0,T]
\end{eqnarray}
with $f(\vecU(t),t) = M^{-1}\spcoper(\df(t),t)$, $M$ being the mass matrix which is in
our case block diagonal or even the identity, depending on the choice of basis
functions. $\vecU(0)$ is given by the projection of $\vecU_0$ onto $\phispace$.

Disregarding the order of the spatial discretization
we use an explicit \textit{Strong Stability
Preserving} Runge-Kutta method (SSP-RK) of third order (\cite{shu:01}). 
Implicit or semi-implicit Runge-Kutta solvers based on a Jacobian-free Newton-Krylov
method (see \cite{knoll:04}) are also available for the proposed DG method 
and implemented in the \dunefem framework.
The results and implementation techniques 
presented in this paper can be applied to explicit, 
implicit, or semi-implicit methods, as long as a 
\textbf{matrix-free} implementation of the discrete operator 
$\spcoper$ is used.

%%%%%%%%%%%%%%%%%%%%%%%%%%%%%%%%%%%%%%%%%%%%%%%
%%%%%%%%%%%%%%%%%%%%%%%%%%%%%%%%%%%%%%%%%%%%%%%

